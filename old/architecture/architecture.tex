\documentclass[12pt,a4paper,notitlepage]{article}
\usepackage[utf8]{inputenc}
\usepackage[a4paper,textwidth=17cm, top=2cm, bottom=3.5cm]{geometry}
\usepackage[T1]{fontenc}
\usepackage{natbib}
\usepackage{ngerman} 
\usepackage{amssymb,amsmath}
\usepackage{listings}
\usepackage{eurosym}
\usepackage{wasysym}
\usepackage{amsthm}
\usepackage{tabularx}
\usepackage{multirow}
\usepackage{multicol}
\usepackage{tikz}
\usepackage{hyperref}
\usepackage{tabularx}
\usepackage{commath}
\usepackage{subfigure}
\usepackage[pdftex]{graphicx,color}
\usepackage{epstopdf}
\newcommand{\re}{\text{Re}}
\newcommand{\im}{\text{Im}}
\newcommand{\foonote}[1]{\footnote{#1}}
\newcommand{\degree}{\ensuremath{^\circ}}
\author{Sebastian Götte {\texttt<securehid@jaseg.net>}}
\title{SecureHID}
\subtitle{Hardening the USB input stack in virtualized environments}
\date{August 12 2018}
\begin{document}
\maketitle

\section{Introduction}
\subsection{Human input devices in a modern desktop system's Trusted Computing Base}
Security in modern computer systems is a complex topic with many practical intricacies. Despite decades-long efforts to
increase the security of end-user systems modern systems provide ample room for catastrophic failure. While some modern
technologies such as web browsers have become hardened to a degree that their user cannot unwittingly fully compromise
their system many other systems such as email have not progressed as much. Today, still, the opening of a seedy email
attachment will in general suffice to cause a full compromise of the user's system. A few dialog boxes and warning
messages have been added but the user experience of causing full compromise is still fundamentally the same only with
the number of clicks now being some four instead of two.

Various architectural solutions to this problem have been proposed, the most promising of which being based on radical
compartmentalization. Smartphone operating systems such as Apple's iOS or Google's Android are single-user systems with
strict application-based compartmentalization. In this paper however we will focus on desktop operating systems due to
their more critical role. Probably the most advanced hardened desktop operating system currently publicly available is
QubesOS. QubesOS uses the Xen hypervisor to enforce strict compartmentalization by usage domain and has been endorsed by
several respected individuals and organizations.

Any system like QubesOS is faced by the fundamental problem of reducing the system's trusted computing base and thereby
reducing attack surface while staying usable in everyday work. A computer bolted to the foundation of a bank vault
guarded by ninjas and having no input devices and no network connections would be as secure as it would be practically
useless.

A particular point of contention in the trusted computing base of any secure system is human input devices. Since
they're used for any authentication and authorization as well as the issuance of any user commands, human input
devices--not some part of the CPU or some trusted platform module--are the single most privileged component of any
system. An attacker in control of the input device or any upstream part of the input stack can observe passwords and
emulate user input amounting to full control of the machine.

\subsection{Attack surface in reasonably secure systems}
\begin{figure}
\tikzstyle{block} = [rectangle, draw, text centered, minimum height=4em]
\begin{tikzpicture}[node distance=2cm, auto]
	\node[block](matrix){Key matrix}
	\node[block](hidctrl){Keyboard controller}
	\node[block](hubs){USB hubs}
	\node[block](roothub){USB host controller}
	\node[block](pcie){PCIe bus}
	\node[block](sys-usb-kernel){USB VM kernel}
	\node[block](sys-usb-agent){USB VM userspace agent}
	\node[block](dom0){dom0 agent}
\end{tikzpicture}
\label{qubes-hid-stack}
\caption{The USB HID input stack in a QubesOS setup}
\end{figure}

Figure \ref{qubes-hid-stack} gives an overview of the components of the USB input stack on a QubesOS system up to dom0,
i.e. the hypervisor. Once an input event arrives at the hypervisor it is propagated back down through a pair of agents
into the domain that currently has keyboard focus.

The QubesOS-specific parts of this stack (the event proxies forwarding the event from the USB VM into dom0 and further
into the target VM) have been designed with security in mind. Their implementations are well-reviewed and their
interfaces have deliberately been kept as simple as possible to reduce the attack surface. Clean design on the part of
QubesOS allows for a high degree of trust into these interfaces. In contrast to this, most of the practical attack
surface in this stack lies on both ends of the physical USB interface. On the one hand, USB is expressly designed to
allow for hot-plugging and online re-enumeration of devices which means any device plugged in to any USB port could
potentially masquerade as a keyboard. This is a very large problem in case of physical access to the device but can also
become a problem for remote attackers gaining some degree of local privilege. With rare exceptions USB firmware
programmers do not recognize the USB interface as a potential target for attack which means an attacker with access to
one USB device can potentially compromise this USB device as part of a larger attack.

Issues like these can in part be mitigated with host-based filtering, such as explicit whitelisting of physical USB
ports for HID devices. In this case, however, the USB driver stack of the linux kernel running the USB VM remains as an
attack surface.

A possible secure solution for this problem would be to completely separate security-critical USB devices such as
keyboard and mouse from everything else. A practical implementation of this would require two separate USB host
controllers in two separate PCIe devices attached to two separate USB VMs, one of which has HID privileges and
everything but the HID drivers blacklisted. This approach has two primary drawbacks. One, it only works in desktop
computers and not in laptops, which often only have a single USB controller soldered onto the mainboard with the input
devices hard-wired into the system. Two, this approach incurs a very large overhead of an entire separate PCIe device
and VM just for HID stack isolation.

In the following sections we will explore a way of securing the USB HID stack at the example of QubesOS without
modifications to the computer itself. Section \ref{commodityhardware} will give an overview over existing secure input
solutions and list some of the challenges to be overcome.

\section{Synthesizing a secure input device from commodity hardware}
\label{commodityhardware}
\subsection{The state of technology}
% FIXME: Definition for "secure input device"
There is two widespread uses of secure input devices in everyday systems. One are the keyboards used for PIN entry on
ATMs and card payment terminals. ATM keyboards form a system that can be described as a Hardware Security Module (HSM)
with buttons. They have their own buffer batteries for always-on active tamper detection. They incorporate security
meshes to form a manipulation-proof security envelope as is usual for other HSMs. Finally, they contain cryptographic
key material to encrypt and authenticate any user input to prevent a manipulated ATM from recording PINs. Though it
provides a reasonable level of security a solution like this is unworkable for everyday use with a computer. A
specialized keyboard built like a HSM would be both very expensive and exceedingly unpopular with users, who in many
cases have very strong preferences regarding their input devices.

Another widespread application of a secure input device is TAN generators used with some electronic payment cards as
part of a ``ChipTAN'' scheme. These devices contain a battery, a small numeric keypad and a small display. Their intent
is to provide a secure channel for transaction confirmation in online banking or online shopping irrespective of the
security of the host machine. The system works by a trusted server generating a challenge for each transaction that is
entered into the TAN generator along with a smartcard. The smartcard uses the TAN generator's keypad and display to ask
the user for confirmation, and in case of confirmation generates a 6-digit response code. The response code is entered
by the user and sent to the trusted server who validates it and executes the transaction.

A scheme like this might work for authorization of infrequent but dangerous actions but fails to work in everyday use.

\subsection{Requirements to a secure input device in the QubesOS setting}
A secure input device in the QubesOS setting has to provide three general characteristics to be useful.
\begin{enumeration}
\item \emph{Authentication} means that only actual input the user gave is accepted, and there is no way for an adversary
	to forge malicious input. Input may neither be re-ordered nor suppressed. Only a complete denial of service is
	acceptable in an attack scenario as it will alert the user that things are amiss.
\item \emph{Secrecy} means that an adversary must not be able to learn the contents of the input the user provides. This
	does not cover timing attacks, which unfortunately will always be possible on a low-latency channel such as
	this.
\item \emph{USB compatibility} means that any solution must be compatible with regular HID devices such as keyboards and
	mice. Special hardware may be required, but no modifications of the existing input device are permittable.
\end{enumeration}

\subsection{Attacker model}
As part of our work we consider attacks on the HID stack as shown in figure \ref{qubes-hid-stack}. We ignore any part
upstream of the USB VM as the codebase of QubesOS is already well-reviewed and engineered to a very high standard of
security.

We consider an attacker that has gained full control of the USB VM and any attached devices. We don't consider a
malicious keyboard a threat due to the \emph{USB compatibility} requirement. Our goal is to protect both the host system
as well as the keyboard from malicious action by the attacker.

\subsection{Security maxims}
Our central design criterion is to keep any interfaces between zones of different trust as simple as possible to reduce
attack surface. Complex interfaces inevitably lead to programming errors which in many cases lead to security
vulnerabilities. In particular we observe that on a high-level view in the HID model information flows from the input
device to the host. The only exception from this is the status of a keyboard's indicator LEDs, which is much
lower-bandwidth than the keyboard input data. This means that an appropriately minimalist implementation of our system
can get away with asymmetric interfaces or, if the status LEDs are dispensable, an entirely one-way interface.

\subsection{Usability challenges and their implications}
% ???

\subsection{System overview}
Our system consists of a small device plugged in between a regular USB keyboard or mouse and the host computer. To
accomodate both keyboard and mouse in one device a two-port USB hub is integrated. The device is internally split into
two sides: The secure side facing keyboard and mouse consists of a powerful, USB host-capable micocontroller. The
insecure side facing the host has a less powerful microcontroller that is only USB device-capable. Both are separated
with an isolation barrier and some supply rail decoupling to frustrate potential side-channel attacks.
% Is an isolation barrier really needed here?

The device has three external USB ports: Keyboard, mouse and host. The device contains both a very bright LED and a
buzzer which are used to signify keyslot changes.

A small button hidden in a hole on the device's back side triggers device/host pairing and a large rotary switch on the
top allows manual keyslot selection. The first position of the keyslot switch is used for insecure HID passthrough for
compatibility with legacy systems.

\section{Hardware implementation}

\section{Cryptographic implementation}
\subsection{Security requirements and attack model}
\subsection{Asymmetric, authenticated ECDH key agreement}
\subsection{Key management}

\section{Firmware considerations}

\section{Conclusion}

\bibliographystyle{plain}
\nocite{*}
\bibliography{overview}

\end{document}
